\chapter{Thesis organization}
This thesis presents the measurement of $W\to l\nu$ and $Z\to ll$ cross-sections in electron and muon channels using 2.76 TeV data collected by \atlas experiment.

The thesis is organized in three parts. The theoretical basis is described in part 1. The experimental input and software organization is explained in the part 2. The cross-section measurement performed by the author is described in part 3. The results and its interpretation via parton density functions is presented in a final part.

The work presented was performed within the ATLAS collaboration. All plots in the thesis were produced by the author, unless it is referenced otherwise. 

The theoretical input as presented in part 1 consist of the following chapters:
\begin{description}
\item [Chapter 2] \textbf{Theoretical introduction}, contains brief overview of the current status of Standard Model, the proton structure and theory of W and Z bosons in pp collisions. The cross-sections predictions at NNLO order are presented.
\item [Chapter 3] \textbf{Methodology}, describes procedures of the cross-section measurement and calculation of their ratios and also methods of PDF extractions using, in addition to other measurement the cross-sections provided by this analysis.
\end{description}

The experimental setup described in the part 2 in the following chapters:
\begin{description}
\item [Chapter 4] \textbf{The LHC and ATLAS experiment} gives an overview of the LHC accelerator complex and its experiments. The \atlas detector, used to collect data for this analysis. is shortly discussed.
\item [Chapter 5] \textbf{Event reconstruction} contains the detailed description of the event reconstruction. The study of missing transverse energy reconstruction algorithm, done by author, is presented.
\item [Chapter 6] \textbf{Monte-Carlo} provides an information of Monte Carlo production steps and a short description of generators used in this analysis.
\item [Chapter 7] \textbf{Frozen showers} gives a description of Frozen Showers method used for fast Monte Carlo simulation. The machine learning procedure for optimization of this method, made by author, is presented.
\item [Chapter 8] \textbf{Data and Monte Carlo samples} describes data and Monte Carlo samples, used in the analysis.
\end{description}

The following chapters present work, done by the author, unless stated otherwise:
\begin{description}
\item [Chapter 9] \textbf{Event selection} gives a set of selection criteria used to derive $W\to l\nu$ and $Z\to ll$ in collected data
\item [Chapter 10] \textbf{Monte-Carlo corrections} presents the correction, applied to Monte Carlo in order to gain a better data vs Monte Carlo agreement. The correction factors have been derived by the performance group, except for muon trigger scale factors, which are determined by the author.
\item [Chapter 11] \textbf{Hadronic recoil calibration} describes a method of missing transverse energy calibration in 2.76 GeV data and methods of the corresponding uncertainty determination.
\item [Chapter 12] \textbf{Background estimation} provides a description of main backgrounds, that can pass the selection criteria and techniques of their contribution estimation
\item [Chapter 13] \textbf{Control distributions} shows the comparison of different data and Monte Carlo simulation distributions for all analyses, covered in this thesis.
\item [Chapter 14] \textbf{Uncertainties of the cross-section measurements} presents main sources of experimental and theoretical uncertainties and gives methods of their propagation to the final cross-sections and their ratios.
\item [Chapter 15] \textbf{Results of the cross-section measurements} presents the results of the cross-section measurement for $W\to l\nu$ and $Z\to ll$ in electron and muon channels separately. The results have been used to test the lepton universality, predicted in the Standard Model.  Also. the combined cross-sections, their ratios are shown. The effect of these measurements on the PDF distributions is estimated.
\end{description}