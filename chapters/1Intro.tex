\chapter{Thesis organization}
This thesis presents the measurement of $W$ and $Z$-boson production cross sections in  the electron and muon channels using 2.76 TeV  p-p data collected by the \atlas experiment in 2013.

The thesis is organized into three main parts. The theoretical basis is described in part 1. The experimental input and software organization is explained in part 2. The measurements performed by the author are described in part 3. The results and its interpretation in the context of parton density functions are presented in a part 3.

The  presented work was performed within the ATLAS collaboration. All plots in this thesis were produced by the author unless it is referenced otherwise. 

The theoretical input as presented in part one consist of the following chapters:
\begin{description}
\item [Chapter 2] \textbf{(Theoretical introduction)} contains a brief overview of the current status of the Standard Model, the proton structure and the theory of the W and Z bosons in pp collisions. The production cross sections predictions are presented;
\item [Chapter 3] \textbf({Methodology)} describes the procedures of the cross section measurement and the calculation of their ratios and also methods of extractions of parton density functions (PDF).
\end{description}

The experimental setup is described in part two in the following chapters:
\begin{description}
\item [Chapter 4] \textbf{(The LHC and the ATLAS experiment)} gives an overview of the Large Hadron Collider (LHC) accelerator complex and its experiments. The \atlas detector used to collect data for this analysis is shortly discussed;
\item [Chapter 5] \textbf{(Event reconstruction)} contains the detailed description of the physics objects reconstruction. The study of missing transverse energy reconstruction algorithm, performed by author, is presented;
\item [Chapter 6] \textbf{(Monte Carlo)} describes the Monte Carlo (MC) production steps and contains short description of generators used in this analysis;
\item [Chapter 7] \textbf{(Frozen Showers)} gives a description of so-called Frozen Showers method used for fast MC simulation at the \atlas experiment. The machine learning procedure for optimization of this method, developed by author, is presented;
\item [Chapter 8] \textbf{(Data and Monte Carlo samples)} describes experimental data and simulated samples, used in the analysis.
\end{description}

\newpage

The following chapters present work, done by the author, unless stated otherwise:
\begin{description}
\item [Chapter 9] \textbf{(Event selection)} gives a set of selection criteria used to derive $W$ and $Z$ boson candidate events in collected data sample;
\item [Chapter 10] \textbf{(Monte Carlo corrections)} presents the corrections applied to the simulated events, which are required to improve data to MC agreement. The correction factors have been derived by the \atlas performance group, except for the muon trigger scale factors, which are determined by the author;
\item [Chapter 11] \textbf{(Hadronic recoil calibration)} describes a method of missing transverse energy calibration in the data and methods for the corresponding uncertainty determination;
\item [Chapter 12] \textbf{(Background estimation)} provides a description of the main background processes and the techniques to estimate their contributions;
\item [Chapter 13] \textbf{(Uncertainties in the cross section measurements)} presents the sources of experimental and theoretical uncertainties and shows the methods of their propagation to the final results;
\item [Chapter 14] \textbf{(Control distributions)} shows the comparison of different distributions between data and MC simulation;
\item [Chapter 15] \textbf{(Results)} presents the results of the measurements of production cross section for $W$ and $Z$-bosons in electron and muon channels. The results have been used to test the lepton universality. Moreover, the combined cross sections and their ratios are shown. The impact of these measurements on the PDF distributions is estimated.
\end{description}
