\chapter{Uncertainties in the cross-section measurement}\label{chap:Unc}
\minitoc
Cross-section measurement relies on theoretical models and corrections, used in Monte-Carlo. Thus, their intrinsic uncertainties should be propagated to a final result. This chapter discusses main methods of uncertainties measurements and sources on $C_{W,Z}$ and $A_{W,Z}$ correction factors. 

\section{Methods of uncertainties propagation}
The offset method changes a correction by a $\pm 1\sigma$ of it's systematic uncertainty. The contribution of each correction's uncertainty on the observable (e.g. $C_{W,Z}$, $A_{W,Z}$ or a cross-section) is taken as a symmetric approximation:
\begin{equation}
U_i^{offset}=\frac{\sigma_{i}^{up}-\sigma_{i}^{down}}{2},
\end{equation}
where $\sigma_{i}^{up(down)}$ - the change in a observable due to the shift of the correction on $\sigma$ up or down. 

Another method used for a uncertainties propagation is a toy MC method, that uses a pseudo experiments with modified input corrections. For a scale factors binned $p_T$ and $\eta$ uncertainties inside each bin can be divided to a correlated  and uncorrelated systematic components and statistical error. For each pseudo-experiment, a table of new scale factors is filled, where inside each bin a scale factor is randomly varied as:
\begin{equation}\label{eq:ToyMethod}
SF_{i}^{Toy_{n}} = SF_{i}+ Gauss(0,\Delta SF_{i} ^{uncorr+stat}) + \sum \Delta SF_{i} ^{corr} \cdot Gauss(0, 1),
\end{equation}
where $SF_{i}^{Toy_{n}}$ is a new scale factor in i-th bin, $\Delta SF_{i} ^{uncorr+stat}$ - is the quadratic sum of uncorrelated and statistical errors and $\Delta SF_{i} ^{corr}$ is a correlated error.

The overall effect on a observable is calculated as a standard deviation of the values in a pseudo-experiments:
\begin{equation}\label{eq:ToyError}
U_{i}=\sqrt{\frac{\sum_{Toy_n=1}^{N} \sigma^2_{i}} {N} + \frac{\sum_{Toy_n=1}^{N} \sigma^2_{i}} {N} }
\end{equation}
The number N of pseudo experiments should be sufficiently large to avoid possible bias in the uncertainty estimation.
 
\section{Experimental systematic uncertainties}
Sources of experimental uncertainties, methods of estimation and their effect on a $C_{W,Z}$ are summarized in a Tab. \ref{tab:Unc}. Systematical errors coming from a hadron recoil calculation are discussed in a Sec. \ref{sec:HadrCalib}. 
\subsection{Electron energy scale and resolution}
Electron energy scale correction, described in Sec. \ref{sec:elecScale} has associated uncertainties coming from <reference>:
\begin{itemize}
\item Statistical component of the scale uncertainty
\item Uncertainty from the possible bias of the calibration method
\item Scale uncertainty from the choice of generator
\item Uncertainty from the presampler energy scale
\item Imperfect knowledge of the material in front of EM calorimeter.
\end{itemize}
The uncertainty contribution from each component is estimated using offset method. The total energy scale uncertainty is the quadratic sum of the components <reference>. 

\subsection{Muon energy scale and resolution}
\begin{itemize}
\item MS modelling
\item ID modelling 
\item overall scale
\end{itemize}
The uncertainty contribution from each component is estimated using offset method. The total energy scale uncertainty is the quadratic sum of the components

\subsection{Muon and electron efficiency toy Monte-Carlo}
 In case of 2.76 TeV analysis scale factor errors are considered to be enlarged for a statistical and uncorrelated components, so correlated error is assumed to be negligible. The toy MC experiments are performed for electron reconstruction, identification and trigger scale factors and muon reconstruction + identification scale factors.  In the current analysis 30 pseudo-experiments are used with a combined toy MC method. 
 Plots for $C_{W}$ effect on a cross-section. 
 Correlation
 
 \newcommand{\rot}{\rotatebox{90}}
\newcommand\tab[1][1cm]{\hspace*{#1}}

\begin{landscape}
\begin{table}[p]
\caption{}
\label{tab:Unc}
\begin{center}
\begin{tabular}{l | c  || c | c || c | c || c | c ||  }
Source of uncertainty & Method & $\delta C_{W} / C_{W} (\%) $ & $\delta C_{W} / C_{W} (\%) $ & $\delta C_{W} / C_{W} (\%) $ & $\delta C_{W} / C_{W} (\%) $ & $\delta C_{Z} / C_{Z} (\%) $ & $\delta C_{Z} / C_{Z} (\%) $\\
 &  & $W^{+}\to e\nu$ & $W^{-}\to e\nu$ & $W^{+}\to \mu\nu$ & $W^{+}\to \mu\nu$ & $Z\to ee$ & $Z\to \mu\mu$ \\
\hline
Electron reconstruction & Toy MC &  \RecEffToyWplusenu  & \RecEffToyWminenu & - & - & \RecEffToyZee  & - \\
Electron identification  & Toy MC &  \IDEffToyWplusenu  & \IDEffToyWminenu &  - & -  & \IDEffToyZee  &  - \\
Electron trigger efficiency & Toy MC &  \TrigToyWplusenu  & \TrigToyWminenu & - & -  & \TrigToyZee  & - \\ 
Muon reco+id & Toy MC &  -  & - & \muIDEffToyWplusmunu & \muIDEffToyWminmunu   & - & \muIDEffToyZmumu \\
Muon trigger  & Offset&  -  & - & \muTrigWplusmunu & \muTrigWminmunu & - & \muTrigZmumu \\
Electron energy scale & Offset &  &  & - & - &  & -\\
\tab - Statistical error & Offset &  &  & - & - &  & -\\
\tab - Bias in method  & Offset &  &  & - & - &  & -\\
\tab - Scale uncertainty &  Offset&  &  & -  & - &  & -\\
\tab - Presampler energy scale & Offset &  &  &-  & - &  &- \\
\tab - Material knowledge &  Offset &  &  & - & - &  &- \\
Electron energy resolution & Offset &\SmearWplusenu  & \SmearWminenu & - & - & \SmearZee  &- \\
Muon energy scale & Offset & - & - & \MuSmearingScaleWplusmunu & \MuSmearingScaleWplusmunu & - & \MuSmearingScaleZmumu \\
Muon energy resolution total & Offset &- & - & Wplusmunu & Wminmunu & - & Zmumu\\ 
\tab - Muon ID energy scale & Offset & - & - & \MuSmearingMSWplusmunu & \MuSmearingMSWminmunu & - & \MuSmearingMSZmumu\\ 
\tab - Muon MS energy scale & Offset & - & - & \MuSmearingIDWplusmunu & \MuSmearingIDWminmunu & - & \MuSmearingIDZmumu\\ 
Hadron recoil scale & Offset &  &  &  &   & - & -\\
Hadron recoil resolution & Offset &  &  &  &  & - & - \\
EWK + $t\bar{t}$ background &  &  &  &  &  &  & \\
QCD  &  &  &  &  & - & - & \\
\hline
PDF error & &  &  &  &  &  & \\
\hline
Total& &  &  &  &  &  & \\
\hline
Statistics & &  &  &  &  &  & \\
\hline
Luminosity & &  &  &  &  &  & \\
\end{tabular}
\end{center}
\end{table}
\end{landscape}


\begin{table}[!tbp]
\caption{}
\label{tab:Unc}
\begin{center}
\begin{tabular}{l | c  || c | c || c | c   }
Source of uncertainty & Method & $\delta C_{W} / C_{W} (\%) $ & $\delta C_{W} / C_{W} (\%) $ & $\delta C_{W} / C_{W} (\%) $ & $\delta C_{W} / C_{W} (\%) $ \\
 &  & $W^{+}\to e\nu$ & $W^{-}\to e\nu$ & $W^{+}\to \mu\nu$ & $W^{+}\to \mu\nu$ \\
\hline
Electron reconstruction & Toy MC & \RecEffToyWplusenu  & \RecEffToyWminenu & - & - \\
Electron identification  & Toy MC & \IDEffToyWplusenu  & \IDEffToyWminenu &  - & - \\
Electron trigger efficiency & Toy MC & \TrigToyWplusenu  & \TrigToyWminenu & - & - \\
Muon reco+id & Toy MC &  -  & - & \muIDEffToyWplusmunu & \muIDEffToyWminmunu \\
Muon trigger  & Offset &   -  & - & \muIDEffToyWplusmunu & \muIDEffToyWminmunu \\
Electron energy scale & Offset &  Wplusenu  & Wminenu & - & -  \\
Electron energy resolution & Offset &\SmearWplusenu  & \SmearWminenu & - & - \\
Muon energy scale & Offset &  -  & - & Wplusmunu & Wminmunu \\
Muon energy resolution & Offset &  -  & - & Wplusmunu & Wminmunu \\
Hadron recoil scale & Offset & Wplusenu  & Wminenu & Wplusmunu & Wminmunu \\
Hadron recoil resolution & Offset & Wplusenu  & Wminenu & Wplusmunu & Wminmunu \\
\hline
Total & &  &  &  &  \\
\hline
\end{tabular}
\end{center}
\end{table}

\begin{table}[!tbp]
\begin{center}
\begin{tabular}{l | c  || c | c || c | c   }
Source of uncertainty & Method & $\delta C_{Z} / C_{Z} (\%) $ & $\delta C_{Z} / C_{Z} (\%) $  \\
 &  & $Z\to ee$ & $Z\to \mu\mu$  \\
\hline
Electron reconstruction & Toy MC & \RecEffToyZee  & Zmumu  \\
Electron identification  & Toy MC &  Zee  & Zmumu  \\
Electron trigger efficiency & Toy MC & \TrigToyZee  & Zmumu  \\
Muon reco+id & Toy MC & Zee  & Zmumu  \\
Muon trigger  & Offset&  Zee  & Zmumu  \\
Electron energy scale & Offset & Zee  & Zmumu  \\
Electron energy resolution & Offset & \SmearZee  & Zmumu  \\
Muon energy scale & Offset &  Zee  & Zmumu  \\
Muon energy resolution & Offset & Zee  & Zmumu  \\
\hline
Total & &  &    \\
\hline
\end{tabular}
\end{center}
\end{table}
\section{Theoretical uncertainty}
\begin{itemize}

\item  Theoretical uncertainties on the predictions are mostly dominated by a imperfect knowledge of the proton PDF's. They are affecting both $A_{W,Z}$ and $C_{W,Z}$. 
Error coming from an arbirtary choice of PDF set is estimated by PDF reweighting <referennce> of original MC generated using <something> to a one of the 4 pdf sets: CT10 <reference>, ATLAS-epWZ12 <reference>, abkm09 <reference> and NNPDF23 <reference>. The error is calculated as a standard deviance for all of the sets.
\item Systematic uncertainty within one pdf set is evaluated using CT10 NLO set. This set contains 52 asosiated error sets, corresponding to a 90\% C.L. limits along 26 eigenvectors. The resulting 52 variation are separetelly added in a quadrature as:
\begin{equation}
\delta_X=\frac{1}{2}\cdot \sqrt{\sum_{i=1}^{N}(X^+-X^-)^2}
\end{equation}
\item The uncertainties arising from the choise of generator and parton showering model are considered small. They can be calculated as a difference in the acceptances  $A_{W,Z}$ for MC samples, generated using same PDF set, but different models for showering and matrix element, namely Powheg + Pythia and Sherpa
\end{itemize}
\section{Correlation between uncertainties}
\subsection{Toy MC correlations}
A correlation coefficient between two observables $o_1$ and $o_2$ can be estimated as:
\begin{equation}
\rho_{12}=\frac{1}{\sigma(o_1)\sigma(o_2)}\cdot \frac{1}{N-1} \sum_{i=1}^N (o^i_1-\bar{o_1}) (o^i_2-\bar{o_2})
\end{equation}
Using cholesky transformation this uncertainty can be propagated to a 3 eigenvectors:


\subsection{Correlations between PDF's eigenvectors}
Correlations for $A_{W,Z}$ and $C_{W,Z}$ for a CT10nlo set have been estimated, since they could affect error on a total W boson cross section measurment and PDF fits. Given two processes X and Y (
\begin{equation}
\delta_{XY}^2 = \delta_X^2+\delta_Y^2+2\delta_X\delta_Y\rho_{XY}
\end{equation} 

\begin{equation}
\rho_{XY}=\frac{1}{4\delta_X\delta_Y}\sum(X^{+}-X^{-})\cdot (Y^{+}-Y^{-})
\end{equation}

\begin{table}[!tbp]
\begin{center}
\begin{tabular}{ c || c | c | c | c }

PDF Set &  $\delta A_{W} / A_{W} (\%) $ & $\delta C_{W} / C_{W} (\%) $ & $\delta A_{W} / A_{W} (\%) $ & $\delta C_{W} / C_{W} (\%) $ \\

 \hline 
  & \multicolumn{2}{c}{$W^{+}\to e\nu$} & \multicolumn{2}{c}{$W^{-}\to e\nu$} \\
  \hline 
 CT10 &  & & & \\
 ATLAS-epWZ12  &  & & & \\
 abkm09  &  & & & \\
 NNPDF23 &  & & & \\
  \hline 
  & \multicolumn{2}{c}{$W^{+}\to \mu\nu$} & \multicolumn{2}{c}{$W^{-}\to \mu\nu$} \\
  \hline 
 CT10 &  & & & \\
 ATLAS-epWZ12  &  & & & \\
 abkm09  &  & & & \\
 NNPDF23 &  & & & \\
\end{tabular}
\end{center}
\end{table}

