\chapter{Selection}\label{chap:EventSelection}
Selection criteria is the set of requirements, that is applied both on data and MC. Analysis depends on a selection, that can separate process of interest (signal) from other processes. For $pp \to W \to e\nu/\mu\nu$ and $pp \to Z/\gamma^* \to ee/\mu\mu$ selection criteria can be divided into 3 groups: data quality, lepton and boson cuts. In this chapter all of them will be discussed and a cut flow presented.
%In this chapter selection criteria for $pp \to W \to e\nu/\mu\nu$ and $pp \to Z/\gamma^* \to ee/\mu\mu$  are presented.  
%you need to add what is truth level in the previous chapter
\section{Data quality cuts}

\begin{table}[h]
    \caption{Analysis selection}
    \label{tab:eventSelection}
    \begin{center}
    \begin{tabular}{ c | c}
        \hline
        \hline
        \multicolumn{2}{c}{Event selection}\\
        \hline
        \multicolumn{2}{c}{Single lepton trigger}\\
        \multicolumn{2}{c}{Good Run List}\\
        \multicolumn{2}{c}{Reject events with LAr errors}\\
        \multicolumn{2}{c}{Number of tracks at primary vertex $\geq$ 3}\\
        \hline
        \hline
        Electron Selection & Muon Selection\\
        \hline

        $P_T>20GeV$ & $P_T>20GeV$\\
        $|\eta|<2.47$ & $|\eta|<2.5$\\
        excluding 1.37<$|\eta|$<1.52 & \\
        OQ cut & staco reconstruction chain \\
        Medium electron identification & Medium muon identification \\
        PtCone20 < 0.1 & PtCone20<0.1 \\

        \hline
        \hline
        W boson selection & Z boson selection \\
        \hline
        EtMiss > 25 GeV &  \\
        $M_T$ > 45 GeV & 66 < $M_{ee}$ < 116 GeV\\
        \hline
        \hline
    \end{tabular}
    \end{center}
\end{table}



Data taking conditions are important in the analysis because of the possible biases. In order to preserve high data quality some events must be rejected. The reason may be unstable beam conditions, disabled parts of the detector or events with too many noisy cells. Number of runs, than can be used for the analysis are stored in a so-called Good Run List (GRL), which in the addition to the run numbers contains information about luminosity blocks. 

Events, where LAr calorimeter was malfunctioning are excluded by LAr quality criteria. Furthermore required to have at least one primary vertex from a hard scattering with at least 2 tracks, that are reconstructed from this vertex. 

Online selection of events is based on a single lepton trigger, depending on a lepton flavor. For electron analysis EF\_e15\_loose1 trigger is used, which records electrons with $E_T$ > 15 GeV. This trigger uses additional "loose" isolation requirements to exclude jets, that are misidentified as electrons. In the muon channel lowest single lepton trigger is EF\_mu10.  It records events with muons $E_T$ >10 GeV.

\section{Lepton quality cuts}

Both analyses use similar selection criteria, applied on a leptons. The leptons must satisfy requirement \ptl > 20 GeV and match to the event trigger.

Electron candidates are required to be within pseudorapidity range \etall < 2.47. The electron candidates found within the transition region between the barrel and the endcap electromagnetic calorimeters, 1.37 < \etall < 1.52, are removed.  Additionally, for a better multijet background rejection  medium identification and \ptcone < 0.1 criterias are applied. The object quality (OQ) cut applied in order to remove events from runs where there dead front end boards in the calorimeter. 

Muons have to satisfy the following criteria: they should be reconstructed by a staco algorithm in a muon spectrometer and fall within range \etall < 2.5 . Set of medium requirements is applied. They must also satisfy \ptcone < 0.1 isolation criteria.

\section{Boson selection}
The events, containing W boson candidates are required to have exactly one lepton, fulfilling the lepton selection. Missing transverse energy, used as a proxy for a neutrino from W decay is required to be \etmiss $>25 GeV$.  The transverse mass, calculated from the lepton and missing transverse energy as in Eq. ~\ref{eq:mtW} has to be bigger than 40 GeV (\mtw > 40 GeV). 

The reconstructed lepton pair in case of Z boson analysis is required to have the invariant mass between 66 and 116 GeV. Both upper and bottom limits allow to exclude regions with high background contamination. 

The full set of cuts is summarized in a Tab. ~\ref{tab:eventSelection}.

\section{Cut flow}

The effect of each selection can be studied using cut-flows, which show the number of events passing each set of cuts in a sequential order. Cut flows for W and Z analysis are shown in a Tab. ~\ref{tab:CutFlowW} and Tab. ~\ref{tab:CutFlowZ} respectively.



