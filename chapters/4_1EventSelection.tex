\chapter{Selection criteria}\label{chap:EventSelection}

Selection criteria defined as a set of requirements used to  separate process of interest (signal) from other processes (background). For $pp \to W \to e\nu/\mu\nu$ and $pp \to Z/\gamma^* \to ee/\mu\mu$ selection criteria can be divided into three main groups: data quality requirements, lepton selection criteria and boson selection criteria. The full set of selection requirements used in the analysis is summarized in Tab.~\ref{tab:eventSelection} and are discussed in this chapter.
%In this chapter selection criteria for $pp \to W \to e\nu/\mu\nu$ and $pp \to Z/\gamma^* \to ee/\mu\mu$  are presented.  
%you need to add what is truth level in the previous chapter
\section{Event selection}

\begin{table}[h]
    \caption{Analysis selection}
    \label{tab:eventSelection}
    \begin{center}
    \begin{tabular}{ c | c}
        \hline
        \hline
        \multicolumn{2}{c}{Event selection}\\
        \hline
        \multicolumn{2}{c}{Single lepton trigger}\\
        \multicolumn{2}{c}{Good Run List}\\
        \multicolumn{2}{c}{Reject events with LAr errors}\\
        \multicolumn{2}{c}{Number of tracks at primary vertex $\geq$ 3}\\
        \hline
        \hline
        Electron Selection & Muon Selection\\
        \hline

        $P_T>20GeV$ & $P_T>20GeV$\\
        $|\eta|<2.47$ & $|\eta|<2.5$\\
        excluding 1.37<$|\eta|$<1.52 & \\
        OQ cut & staco reconstruction chain \\
        Medium electron identification & Medium muon identification \\
        PtCone20 < 0.1 & PtCone20<0.1 \\

        \hline
        \hline
        W boson selection & Z boson selection \\
        \hline
        EtMiss > 25 GeV &  \\
        $M_T$ > 45 GeV & 66 < $M_{ee}$ < 116 GeV\\
        \hline
        \hline
    \end{tabular}
    \end{center}
\end{table}



Data-taking conditions are important in the analysis, so in order to preserve high data quality, the selection criteria are applied. The events with unstable beam conditions, disabled parts of the detector or events with high noise in calorimeter are rejected. Runs, that can be used in the analysis are stored in the so-called Good Run List (GRL).

Events for which LAr calorimeter was malfunctioning are excluded by LAr quality criteria. Furthermore, events are required to have at least one primary vertex formed with at least three tracks. 

Online selection of events is based on single lepton triggers. For electrons EF\_e15\_loose1 trigger is used, which records electrons with $E_T$ > 15 GeV. This trigger uses additional "loose" isolation requirements to exclude jets, that are misidentified as electrons. In the muon channel, the lowest single lepton trigger is EF\_mu10.  It records events containing at least one muon with $P_T$ >10 GeV.




\section{Lepton selection}

Both W and Z-boson analyses use similar offline lepton selection criteria. All leptons must satisfy a requirement, \ptl > 20 GeV. The leptons are required to trigger the event and therefore to be within the distance $\Delta R = \sqrt{\Delta\eta^2+\Delta\phi^2}=0.2$ with the event trigger.

Electron candidates are required to be within pseudorapidity range of \etall < 2.47. Because the precise reconstruction of the electrons is not possible in the transition region between barrel and endcap, the electron candidates reconstructed within the pseudorapidity range 1.37 < \etall < 1.52 are not used.  Additionally, for a better multijet background rejection  medium identification criteria are applied. The object quality (OQ) criteria are also applied to electron candidates in order to remove events from runs where EM calorimeter was malfunctioning.

Muons have to satisfy the following offline selection criteria: they should be reconstructed by a staco algorithm and fall within a range of \etall < 2.5. Additionally, for a better rejection of background, muons are required to be combined, i.e. having a matched track in both ID and MS.

In order to reach a better background rejection, the \textit{isolation criteria} (\ptcone < 0.1) is applied on both electron and muon candidates. These criteria uses the information about ID tracks that fall within a distance of $\Delta R = \sqrt{\Delta\eta^2+\Delta\phi^2}=20$ around the lepton direction. Events with the sum of all tracks momenta (except for the selected lepton)  greater than $0.2\times \ptl$ are excluded from the analysis. 

\section{Boson selection}
Events, containing W-boson candidates are required to have exactly one reconstructed lepton. Missing transverse energy is required to be \etmiss > 25 GeV.  Transverse mass, calculated from the lepton and the missing transverse energy (Eq.~\ref{Eq:mtW}) has to be bigger than 40 GeV.

Events for the Z-boson selection are required to contain exactly two, opposite sign, same flavor lepton candidates.  The invariant mass of the reconstructed lepton pair required to fall within region 66 GeV < $M^{Z}$ < 116 GeV. 

The effect of each selection can be studied using the number of events passing each set of selections in a sequential order (Tab.~\ref{tab:CutFlowW}). The events, fulfilling the all of the W and Z-boson selection requirements are called W and Z-boson candidate events, respectively.

\begin{table}[h]
    \caption{Number of W and Z-boson candidate events in data and signal simulation, remaining after each major requirement. The simulation is normalized to the NNLO cross section shown in Tab.~\ref{tab:Backgrounds}.}
    \label{tab:CutFlowW}
    \begin{center}
    \begin{tabular}{ l | c | c || c | c  }
    \hline
     & \multicolumn{4}{c}{Number of candidates} \\
     Requirements & Data & signal MC & Data & signal MC \\
     \hline
     \hline
    & \multicolumn{2}{c ||}{$W^{+}\to e\nu$} & \multicolumn{2}{c}{$W^{+}\to \mu\nu$}    \\
    \hline
    No selection & \cutFlowTotWplusenuData & \cutFlowTotWplusenuMC  & \cutFlowTotWplusmunuData & \cutFlowTotWplusmunuMC  \\
    Event selection &\cutFlowEventWplusenuData &\cutFlowEventWplusenuMC & \cutFlowEventWplusmunuData & \cutFlowEventWplusmunuMC  \\
    Lepton selection &\cutFlowLeptonWplusenuData  & \cutFlowLeptonWplusenuMC  & \cutFlowLeptonWplusmunuData &\cutFlowLeptonWplusmunuMC \\
    Boson selection & \cutFlowBosonWplusenuData & \cutFlowBosonWplusenuMC  &\cutFlowBosonWplusmunuData &\cutFlowBosonWplusmunuMC \\
    \hline
    \hline
    & \multicolumn{2}{c ||}{$W^{-}\to e\nu$} & \multicolumn{2}{c}{$W^{-}\to \mu\nu$} \\
    \hline
    No selection & \cutFlowTotWminenuData & \cutFlowTotWminenuMC & \cutFlowTotWminmunuData & \cutFlowTotWminmunuMC \\
    Event selection &\cutFlowEventWminenuData & \cutFlowEventWminenuMC  & \cutFlowEventWminmunuData & \cutFlowEventWminmunuMC\\ 
    Lepton selection & \cutFlowLeptonWminenuData & \cutFlowLeptonWminenuMC  & \cutFlowLeptonWminmunuData & \cutFlowLeptonWminmunuMC \\
    Boson selection  & \cutFlowBosonWminenuData &\cutFlowBosonWminenuMC  &\cutFlowBosonWminmunuData &\cutFlowBosonWminmunuMC \\
    \hline
    \hline
        & \multicolumn{2}{c ||}{$Z \to ee$} & \multicolumn{2}{c}{$Z \to \mu\mu$} \\
        \hline
    No selection &  \cutFlowTotZeeData & \cutFlowTotZeeMC & \cutFlowTotZmumuData & \cutFlowTotZmumuMC \\
    Event selection & \cutFlowEventZeeData &\cutFlowEventZeeMC & \cutFlowEventZmumuData & \cutFlowEventZmumuMC \\
    Lepton selection & \cutFlowLeptonZeeData  & \cutFlowLeptonZeeMC & \cutFlowLeptonZmumuData & \cutFlowLeptonZmumuMC \\
    Boson selection & \cutFlowBosonZeeData & \cutFlowBosonZeeMC & \cutFlowBosonZmumuData & \cutFlowBosonZmumuMC \\
    \hline
    \end{tabular}
  \end{center}
\end{table}