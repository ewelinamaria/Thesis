\chapter{Methodology of the measurement}\label{chap:Met}
\minitoc
\section{Cross-Section methodology}

The production cross-section of W and Z bosons in a fiducial region, corresponding to the geometrical acceptance of the detector and of the kinematic selection, measured using the  equation:
\begin{equation}
\sigma^{fid}_{W/Z} = \frac{N-B}{C_{W/Z}L_{int}},
\end{equation}
where 
\begin{itemize}
\item N is the number of candidates measured in a data
\item B is the number of background events
\item $L_{int}$ is the integrated luminosity corresponding to a run selections and trigger requirements
\item $C_{W/Z}$ is a correction factor for an experimental selection and resolution effects.
\end{itemize}

The correction factors $C_{W/Z}$ is calculated from MC for each process and each decay channel separatelly and defined as :
\begin{equation}
C_{W/Z}=\frac{N_{MC, rec}}{N_{MC,gen, cut}}, 
\end{equation}
where $N_{MC, rec}$ are sums of weights of events after simulation, reconstruction and selection, $N_{MC,gen, cut}$ - are taken on the generator level after fiducial cuts. These correction factor are including efficiencies for trigger, reconstruction and identification (see Sec.\ref{sec:Eff}).

In addition, this measurement could be extrapolated to the full phase-space using as:
\begin{equation}
\sigma^{tot}_{W/Z}= \frac{\sigma^{fid}_{W/Z}}{A_{W/Z}}= \frac{N-B}{A_{W/Z}C_{W/Z}L_{int}},
\end{equation}
where $\sigma^{tot}_{W/Z}$ are the total inclusive production cross-section of the W and Z bosons, and $A_{W/Z}$  is an acceptance. 

The $A_{W/Z}$ factor determined from Monte Carlo simulation as: 
\begin{equation}
A_{W/Z}=\frac{N_{MC,gen, cut}}{N_{MC,gen,all}},
\end{equation}
where $N_{MC,gen,all}$ are the sum of weights of all generated MC events. Both $A_{W/Z}$ and $C_{W/Z}$ are defined at the "born level", i.e. before the decay leptons emit photons via QED final state radiation.

\subsection{Fiducial phase-space definition}
The definition of fiducial volume for W bosons in both electron and muon channels is:
\begin{itemize}
\item $P_T^l$ > 20 GeV
\item $|\eta^l|$ < 2.5 
\item $P_T^{\nu}$ > 25 GeV
\item \mtw > 40 GeV
\end{itemize}
where $P_T^l$ and $P_T^{\nu}$ are the charged lepton and neutrino transverse momentums respectivelly, $\eta^l $ is a lepton pseudo-rapidity, and \mtw is the transverse mass, defined as:
\begin{equation}\label{Eq:mtW}
\mtw = \sqrt{ 2 P_T^l \cdot P_T^{\nu} [ 1 - cos (\phi_l - \phi_{\nu})]},
\end{equation}
where $\phi_l$ is an azimutal angle of the charged lepton and $\phi_{\nu}$ is an azimutal angle of neutrino.

For the Z boson measurment fiducial phase-space is defined as:
\begin{itemize}
\item $P_T^l$ > 20 GeV
\item $|\eta^l|$ < 2.5 
\item 66 GeV <$M_{Z}$ < 116 GeV
\end{itemize}
where $M_{Z}$ is an di-lepton invariant mass.

\subsection{Extrapolation to the 13 TeV fiducial phase-space}

Since the  13 TeV inclusive cross-section measurement  uses another definition of the fiducial phase space, it is possible to extrapolate cross-section to a new fiducial phase-space as:
\begin{equation}
\sigma^{fid,13}_{W/Z} = \frac{\sigma^{fid}_{W/Z}}{E_{W/Z}},
\end{equation}
where $E_{W/Z}$ is an additional extrapolation factor:
\begin{equation}
E_{W/Z}=\frac{N_{MC,gen, cut}}{N_{MC,gen, cut^{new}}},
\end{equation}
where $N_{MC,gen, cut^{new}}$ is the sum of the weights of Monte-Carlo events after the new set of cuts on the generator level.

The 13 TeV fiducial phase-space is defined as:
\begin{itemize}
\item $P_T^l$ > 25 GeV
\item $|\eta^l|$ < 2.5 
\item $P_T^{\nu}$ > 25 GeV
\item \mtw > 50 GeV
\end{itemize}
for the W decays and 
\begin{itemize}
\item $P_T^l$ > 25 GeV
\item $|\eta^l|$ < 2.5 
\item 66 GeV <$m_{Z}$ < 116 GeV
\end{itemize}
for the Z measurements.

\section{Ratios calculation}
\section{Results combination}\label{sec:Aver}
\section{PDF fits}