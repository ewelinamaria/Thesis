\chapter{Measurement methodology}\label{chap:Met}
\minitoc

 In this chapter, a methodology of the measurements is given. The first section gives a description of cross section measurement in the different phase space regions. The second section presents the method of linear averaging of the results, used for the combination of electron and muon channel analyses. In the last section, the PDF fit procedure used to put constraints on PDFs is described.
 
\section{Cross section calculation}
Due to the limited geometrical detector acceptance and reconstruction efficiency of the \atlas detector (Sec.~\ref{sec:ATLAS}) the W and Z-boson cross sections cannot be measured in the full phase space. The fiducial cross section, i.e. the cross section limited to a given selection region, is measured, using the formula:
\begin{equation}
\sigma^{fid}_{W/Z} = \frac{N^{W/Z}-B^{W/Z}}{C_{W/Z}L_{int}}=\frac{N^{W/Z}_{sig}}{C_{W/Z}L_{int}},
\end{equation}
where:
\begin{itemize}
\item $N$ is the number of events measured in the data;
\item $B$ is the estimated number of background events;
\item $N^{W/Z}_{sig}=N^{W/Z}-B^{W/Z}$ is the number of the signal events;
\item $L_{int}$ is the integrated luminosity of a dataset;
\item $C_{W/Z}$ is a correction factor that accounts for event selection and detector resolution effects.
\end{itemize}

For each process, the correction factor $C_{W/Z}$ is calculated from the simulation (Chap.~\ref{chap:MC}) as :
\begin{equation}
C_{W/Z}=\frac{N_{MC, rec}}{N_{MC,gen, selection}}, 
\end{equation}
where $N_{MC, rec}$ is a total number of simulated events which pass the final selection requirements after reconstruction, $N_{MC,gen, selection}$ total number of simulated events after fiducial selection on the generator level.

The measurements can be then extrapolated to the full phase space using simulated events:
\begin{equation}
\sigma^{tot}_{W/Z}= \frac{\sigma^{fid}_{W/Z}}{A_{W/Z}},
\end{equation}
where $\sigma^{tot}_{W/Z}$ is the total inclusive production cross section for the W or Z-bosons, and $A_{W/Z}$  is the acceptance factor. 

The $A_{W/Z}$ factor is determined from the simulation simulation using: 
\begin{equation}
A_{W/Z}=\frac{N_{MC,gen, selection}}{N_{MC,gen,all}},
\end{equation}
where $N_{MC,gen,all}$ is a total number of simulated events. Both $A_{W/Z}$ and $C_{W/Z}$ are defined at the "born level", i.e. before the decay leptons emit photons via QED final state radiation.

\subsection{Fiducial phase space definition}
The fiducial region definition corresponds to the analysis selection described in Chap.~\ref{chap:EventSelection}. For W boson measurement it is depicted as:
\begin{itemize}
\item $P_T^l$ > 20 GeV;
\item $|\eta^l|$ < 2.5;
\item $P_T^{\nu}$ > 25 GeV;
\item \mtw > 40 GeV,
\end{itemize}
where $P_T^l$  ($P_T^{\nu}$) is the charged lepton (neutrino) transverse momentum, $\eta^l $ is a lepton pseudo-rapidity, and \mtw is the transverse mass, defined as:
\begin{equation}\label{Eq:mtW}
\mtw = \sqrt{ 2 P_T^l \cdot P_T^{\nu} [ 1 - cos (\phi_l - \phi_{\nu})]},
\end{equation}
where $\phi_l - \phi_{\nu}$ is an azimuthal angle between charged lepton and neutrino.

For the Z-boson production measurement the fiducial phase space is defined as:
\begin{itemize}
\item $P_T^l$ > 20 GeV;
\item $|\eta^l|$ < 2.5;
\item 66 GeV < $M_{Z}$ < 116 GeV,
\end{itemize}
where $M_{Z}$ is a di-lepton invariant mass. 

The differences between analysis selection for the electron and muon channels are neglected in this definition.

\subsection{Extrapolation to the 13 TeV fiducial phase space}

The  inclusive cross section measurements at $\sqrt{s}$=13 TeV use stricter set of analysis criteria and therefore definition of the fiducial phase space. However, it is possible to extrapolate the relevant cross sections to slightly different fiducial phase space using the relation:
\begin{equation}
\sigma^{fid,13}_{W/Z} = \frac{\sigma^{fid}_{W/Z}}{E_{W/Z}},
\end{equation}
where $E_{W/Z}$ is the extrapolation factor, defined as:
\begin{equation}
E_{W/Z}=\frac{N_{MC,gen, selection}}{N_{MC,gen, selection^{new}}},
\end{equation}
where $N_{MC,gen, selection^{new}}$ total number of simulated events after new fiducial selection on the generator level. The extrapolation procedure allows to directly calculate the cross sections ratios for a different $\sqrt{s}$.

The 13 TeV fiducial phase space is defined as:
\begin{itemize}
\item $P_T^l$ > 25 GeV;
\item $|\eta^l|$ < 2.5;
\item $P_T^{\nu}$ > 25 GeV;
\item \mtw > 50 GeV
\end{itemize}
for the W-boson production decays and as:
\begin{itemize}
\item $P_T^l$ > 25 GeV;
\item $|\eta^l|$ < 2.5;
\item 66 GeV <$m_{Z}$ < 116 GeV
\end{itemize}
for the Z-boson measurement.

\subsection{The $W$ boson cross section calculation}\label{sec:Wcs}
In this analysis, the cross sections for the production of W boson calculated by combining $W^{+}$ and $W^{-}$-bosons production cross sections in the following way:
\begin{equation}
\sigma_{W}=\sigma_{W^{+}}+\sigma_{W^{-}} = \frac{1}{L_{int}} \cdot \Big(\frac{N^{W^+}_{sig}}{A_{W^+}C_{W^+}} +  \frac{N^{W^-}_{sig}}{A_{W^-}C_{W^-}} \Big). 
\end{equation}

The absolute systematic uncertainty of this measurement is calculated from uncertainties of $W^{+}$ and $W^{-}$ cross sections as:
\begin{equation}
(\delta X_{W})^2 = (\delta X_{W^{+}})^2+(\delta X_{W^{-}})^2 + 2\cdot \rho^{X}_{W^+W^-}\delta X_{W^{+}}\delta X_{W^{-}},
\end{equation}
where $\delta X$ is a systematic component on the cross section and $\rho^{X}_{W^+W^-}$ is a correlation between $W^{+}$ and $W^{-}$-bosons production for this component, which is estimated in Chap.~\ref{chap:Unc}. 

\section{Averaging of the results}\label{sec:Aver}

The SM predicts the same branching ratios for the leptonic decays of W and Z bosons. Therefore, it is possible to combine the measurements in the electron and muon channels into a single cross section. This analysis uses the standard tool for averaging of the measurements (called Havereger), that was originally developed  for the HERA experiment\cite{HERADIS}. The tool uses a method of linear averaging, which is described below.

For a given observable, the probability density function for a "true" value $m$ to get a value $\mu$ with uncertainty $\Delta$ in the measurement is:
\begin{equation}
P(m)=\frac{1}{\sqrt{2\pi\Delta}}exp\Big(-\frac{(m-\mu)^2}{2\Delta^2}\Big).
\end{equation}
 The corresponding $\chi^2$ function, that describes the goodness of fit of an observed values to a "true" value, calculated as:
\begin{equation}\label{eq:chi2Ave}
\chi^2(m) = \frac{(m-\mu)^2}{\Delta^2}.
\end{equation}

In case of N statistically-independent measurements, one has:
\begin{equation}
P(m)\propto \prod_{i=0}^{N} exp\Big(-\frac{(m-\mu_i)^2}{2\Delta_i^2}\Big),
\end{equation}
which corresponds to the:
\begin{equation}
\chi_{sum}^2(m) = \sum_{i} \chi^2_i = \sum_{i} \frac{(m-\mu_i)^2}{\Delta_i^2}, 
\end{equation}
that can be rewritten in the form of the Eq.~\ref{eq:chi2Ave}:
\begin{equation}\label{eq:chi2Sum}
\chi_{sum}^2(m) =\frac{(m-\mu_{ave})^2}{\Delta_{ave}^2}+\chi^{2}_{0}, 
\end{equation}
where $\mu_{ave}$ and $\Delta_{ave}$ are the average value and its uncertainty respectively.  These values are found by minimizing $\chi^{2}_{sum}$ with respect to $m$.  The value $\chi^2_0$ indicate a consistency of the measurements and should be $\chi^2_0/N \approx 1$.

Systematic uncertainties can be treated as a result of imperfect experiment (e.g. due to its calibration) and are added to the \chiD function:
\begin{equation}
\chiD_{syst}(\alpha) = \frac{\alpha-\alpha_0}{\Delta_{\alpha}^2}\equiv b^2,
\end{equation}
where $\alpha$ is a "true" value of parameter (what is measured at the experiment as $\alpha_0$ with uncertainty $\Delta_{\alpha}$). The nuisance parameter $b$ corresponds to a coherent change of the measurements due to systematic effects $\mu_i \to \mu_i+\mathbf{bF_i}$. 

Using this nuisance parameters representation, Eq.~\ref{eq:chi2Sum} can be rewritten in a more general way:
\begin{equation}\label{Eq:ChiDParam}
\chi^{2}_{sum}(\mathbf{m},\mathbf{b})=\sum_{i} \frac{(m-\mu_i-\sum_j\Gamma_i^jb_j)^2}{\Delta_i^2}+\sum_j b_j^2, 
\end{equation}
where:
\begin{itemize}
\item $i$ runs over all measured values used in averaging;
\item $\mathbf{b}$ is the vector of nuisance parameters, $b_j$, corresponding to each source of systematic uncertainties;
\item $\Gamma_i^j$ is the absolute correlated systematic uncertainty;
\item $\Delta_i^2$ is the uncorrelated (statistical) uncertainty.
\end{itemize}


\section{Calculation of the cross section ratios}\label{sec:Rat}

The calculation of the cross section ratios is a powerful tool for testing theory predictions, due to the cancellation of the correlated uncertainties. The ratio for two cross section measurements ($\sigma_i$ and $\sigma_j$) is calculated as:
\begin{equation}
R_{i,j}=\frac{\sigma_i}{\sigma_j}=\frac{\frac{N^{i}_{sig}}{A^{i}_{W/Z}C^{i}_{W/Z}L_{int}}}{ \frac{N^{j}_{sig}}{A^{j}_{W/Z}C^{j}_{W/Z}L_{int}}}
=\frac{N^{i}_{sig}}{N^{j}_{sig}}\cdot \frac{A^{j}_{W/Z}}{A^{i}_{W/Z}} \cdot \frac{C^{j}_{W/Z}}{C^{i}_{W/Z}}=\frac{N^{i}_{sig}}{N^{j}_{sig}}\cdot A_{i/j} C_{i/j}.
\end{equation}
This means that $R_{i,j}$ does not depend e.g. on the integrated luminosity and its uncertainty.

The relative uncertainty on $R_{i,j}$ can therefore be obtained by taking into account the correlations between two measurements:
\begin{equation}
\Big(\frac{\delta R_{i,j} }{R_{i,j}}\Big)^2=\Big(\frac{\delta N^{i}_{sig}}{N^{i}_{sig}}\Big)^2+\Big(\frac{\delta N^{j}_{sig}}{N^{j}_{sig}}\Big)^2+\Big(\frac{\delta A_{i/j}}{A_{i/j}}\Big)^2+\Big(\frac{\delta C_{i/j}}{C_{i/j}}\Big)^2.
\end{equation}

Uncertainties on the first two terms are considered to be uncorrelated, while the uncertainties on $A_{i,j}$ and $C_{i,j}$ are derived using the following error propagation formula:
\begin{equation}
\Big(\frac{\delta X_{i/j} }{X_{i/j}}\Big)^2= \Big(\frac{\delta X_{i} }{X_{i}}\Big)^2+\Big(\frac{\delta X_{j} }{X_{j}}\Big)^2-2\rho_{ij}\frac{\delta X_{i}}{X_{i}}\frac{\delta X_{j}}{X_{j}},
\end{equation}
where X is a relevant systematic component and $\rho_{ij}$ is a correlation between two measurements. The estimation of correlation parameters is discussed in details in Chap.~\ref{chap:Unc}.


\section{Estimation of the parton density functions}\label{sec:PDFFit}

As already mentioned in Chap.~\ref{chap:Theory}, the parton density functions (PDFs) cannot be calculated perturbatively in QCD. However, they can be calculated from the global fit to the experimental data. In this thesis the xFitter program\cite{Xfitter} is used to determine the proton PDFs. 

The PDFs are defined at a given starting scale $Q^2_0$ and then evoluted to a needed scale using the DLGAP equations. There are different parameterizations used for the PDFs at a starting scale. A very standard form uses a simple polynomial for interpolation between low- and high-$x$ regions:
\begin{equation}
xf(x; Q^2_0) = A x^{B} (1-x)^{C} P_{i}(x),
\end{equation}
where $P_{i}$ is a polynomial of a given order. In the analysis presented in Chap.~\ref{chap:Res} a standard HERA parameterization is used, implementation is discussed in details below.

There are 11 different partons to consider, however heavy quark ($c$ and $b$-quarks) distributions can be determined perturbatively. This leaves at least 7 independent combinations. The parameterized PDFs are the valence distributions ($xu_{v}$ and $xd_{v}$), the gluon distribution )$xg$), and the u-type ($x\bar{U}$) and d-type ($x\bar{D}$) sea, where:
\begin{center}
$xu_{v} = xu - x\bar{u}\, $;  $xd_{v} = xd-x\bar{d}\, $;   $x\bar{U}=x\bar{u}\, $;  $x\bar{D}=x\bar{d}+x\bar{s}\, $. 
\end{center}
For the strange quark distributions, it is assumed that at a given starting scale $Q^2_0$:
\begin{equation}
xs = x\bar{s} = f_sx\bar{D},
\end{equation} 
where $f_s = 0.31$ - is a strange quark fraction which is chosen to match the experimental data. 

The following functional form is used for PDFs parameterization:
\begin{equation}
xf(x; Q^2_0) = A x^{B} (1-x)^{C} (1+Dx+Ex^2),
\end{equation}
where $A_{uv}$, $A_{dv}$ and $A_g$ are constrained by the number sum-rules and the momentum sum-rule, and the parameters $B_{\bar{U}}$ and $B_{\bar{D}}$ are set equal, so there is a single parameter for the sea distributions\cite{xfittManual}.

Similarly to averaging procedure from Sec.~\ref{sec:Aver}, the nuisance parameters representation of the experimental systematic uncertainties is used in the PDF fiting procedure:
\begin{equation}
\mu_i = m_i(\mathbf{p})+r_i\sigma_i + \sum_{\alpha=1}^{N_{syst}} \Gamma_{\alpha}^{i}b_{\alpha}, 
\end{equation}
where:
\begin{itemize}
\item $m_i(\mathbf{p})$ is the true value of a given observable, that depends on set of parameters $\mathbf{p} = (p_1, p_2,...)$;
\item $\mu_i$ is the value, observed in the experiment;
\item $\sigma_i$ are the statistical and systematic uncertainties;
\item $\Gamma_{\alpha}^{i}$ is the sensitivity of i-th measurement to the correlated systematic source $\alpha$;
\item $r_i$ are the normally distributed random variables;
\item $b_{\alpha}$ are the nuisance parameters associated with the corresponding experimental systematic uncertainty.
\end{itemize}

The simple parameterization form of \chiD defined in Eq.~\ref{Eq:ChiDParam} can be also used. This equation can be rewritten by introducing a Poisson distribution for the statistical uncertainty:
\begin{equation}
\chi^{2}_{sum}(\mathbf{m},\mathbf{b})=\sum_{i} \frac{(m-\mu_i-\sum_j\Gamma_i^jb_j)^2}{\delta^{2}_{i,stat}\mu^{i}m^{i}\prod_{\alpha}exp(-\gamma_{\alpha}^ib_{\alpha})}+\sum_{\alpha} b_{\alpha}^2.
\end{equation}

The PDF uncertainties are estimated by varying the data points within their statistical and systematic uncertainties using a Monte Carlo technique. For each data set, a QCD fit is performed to extract a given PDF set. 

The effect of adding new data to the PDF evaluation procedure can be obtained using the profiling procedure\cite{PDFProf}. It is preformed using the following representation of the \chiD function (with both theoretical and experimental uncertainties included):
\begin{equation}
\chi^{2}(\mathbf{m},\mathbf{b_{exp}}, \mathbf{b_{th}})=\sum_{i} \frac{(m-\mu_i-\sum_\alpha \Gamma_\alpha^{exp}b_{\alpha, exp}+\sum_\beta \Gamma_\beta^{th}b_{\beta, th})^2}{\delta^{2}_{i,stat}\mu^{i}m^{i}\prod_{\alpha}exp(-\gamma_{\alpha}^ib_{\alpha})}+\sum_{\alpha} b_{\alpha, exp}^2 + \sum_{\beta} b_{\beta, th}^2,
\end{equation}
where $\mathbf{b_{exp}}$ and $\mathbf{b_{th}}$ are the vectors of correlated experimental and theoretical uncertainties respectively. The sensitivity of the measurements is splitted into two experimental and theoretical components: $\Gamma_\alpha^{exp}$ and $\Gamma_\beta^{th}$. This \chiD function is represented as a system of linear equations, that can be minimized iteratively, allowing to determine a shifted PDF parameters. Then, the value of \chiD at its minimum provides a compatibility of the new measurement with a given PDF set.
